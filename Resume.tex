\documentclass[a4paper]{article}
    \usepackage{hyperref}
    \usepackage{fullpage}
    \usepackage{amsmath}
    \usepackage{amssymb}
    \usepackage{textcomp}
    \usepackage[utf8]{inputenc}
    \usepackage[T1]{fontenc}
    \textheight=5in
    \pagestyle{empty}
    \raggedright
    \usepackage[left=0.4in,right=0.4in,bottom=0.3in,top=0.5in]{geometry}
\usepackage{etoolbox,refcount}
\usepackage{multicol}


\newcounter{countitems}
\newcounter{nextitemizecount}
\newcommand{\setupcountitems}{%
  \stepcounter{nextitemizecount}%
  \setcounter{countitems}{0}%
  \preto\item{\stepcounter{countitems}}%
}
\makeatletter
\newcommand{\computecountitems}{%
  \edef\@currentlabel{\number\c@countitems}%
  \label{countitems@\number\numexpr\value{nextitemizecount}-1\relax}%
}
\newcommand{\nextitemizecount}{%
  \getrefnumber{countitems@\number\c@nextitemizecount}%
}
\newcommand{\previtemizecount}{%
  \getrefnumber{countitems@\number\numexpr\value{nextitemizecount}-1\relax}%
}
\makeatother    
\newenvironment{AutoMultiColItemize}{%
\ifnumcomp{\nextitemizecount}{>}{3}{\begin{multicols}{2}}{}%
\setupcountitems\begin{itemize}}%
{\end{itemize}%
\unskip\computecountitems\ifnumcomp{\previtemizecount}{>}{3}{\end{multicols}}{}}


    %\renewcommand{\encodingdefault}{cg}
%\renewcommand{\rmdefault}{lgrcmr}

\def\bull{\vrule height 0.7ex width .7ex depth -.1ex }

% DEFINITIONS FOR RESUME %%%%%%%%%%%%%%%%%%%%%%%
\hypersetup{
    colorlinks=true,
    linkcolor=black,
    filecolor=magenta,      
    urlcolor=black,
    pdftitle={Eishkaran's Resume},
    pdfpagemode=FullScreen,
    }

\newcommand{\area} [2] {
    \vspace*{-9pt}
    \begin{verse}
        \textbf{#1}   #2 
    \end{verse}
}

\newcommand{\lineunder} {
    \vspace*{-8pt} \\
    \hspace*{-18pt} \hrulefill \\
}

\newcommand{\header} [1] {
    {\hspace*{-18pt}\vspace*{6pt} \textsc{#1}}
    \vspace*{-6pt} \lineunder
}

\newcommand{\employer} [3] {
    { \textbf{#1} (#2)\\ \underline{\textbf{\emph{#3}}}\\  }
}

\newcommand{\contact} [3] {
    \vspace*{-10pt}
    \begin{center}
        {\Huge \scshape {#1}}\\
        #2 \\ #3
    \end{center}
    \vspace*{-8pt}
}

\newenvironment{achievements}{
    \begin{list}
        {$\bullet$}{\topsep 0pt \itemsep -2pt}}{\vspace*{4pt}
    \end{list}
}

\newcommand{\schoolwithcourses} [4] {
    \textbf{#1} #2 $\bullet$ #3\\
    #4 \\
    \vspace*{5pt}
}

\newcommand{\school} [4] {
    \textbf{#1} #2 $\bullet$ #3\\
    #4 \\
}
% END RESUME DEFINITIONS %%%%%%%%%%%%%%%%%%%%%%%

    \begin{document}
\vspace*{-40pt}

    

%==== Profile ====%
\vspace*{-9pt}
\begin{center}
	{\Huge \scshape {Eishhkaran Singh}}\\
	\vspace{1mm}
	India $\cdot$ eishkaransingh@gmail.com $\cdot$ +91 7087694506 $\cdot$ \href{https://github.com/EISHKARAN}{GitHub} $\cdot$ \href{https://www.linkedin.com/in/eishkaran-singh/}{Linkedin} \\
\end{center}

% \begin{center}
% Dedicated Data Scientist with hands-on experience in data analysis, machine learning, and programming. Skilled in using various tools and technologies to extract valuable insights from complex data sets.\\
% \end{center}
\vspace{0.25mm}
%==== Education ====%
\header{Education}
\vspace{0mm}
\textbf{Thapar Institute of Engineering and Technology}\hfill Patiala\\
BTech Computer  Engineering \hfill September 2021 - Present\\
\vspace{0.5mm}
\textbf{Indian Institute of Technology}\hfill Madras\\
Bs Data Science and Applications \hfill January 2022 - Present\\
\vspace{1mm}

 
% \header{Position of Responsibility }
% \vspace{0mm}

% \textbf{GNU/Linux Users Club} \hfill IIIT Bhopal \\
% \textit{Club Secretary} \hfill August 2022 - Present\\
% \vspace{-2.5mm}
% \begin{itemize} \itemsep 1pt
% 	\item Using MacOS as my main distribution but I also have experience in most Debian based (Ubuntu), Fedora and Arch based Distributions
% 		\item Guiding batch mates to install and use Linux distributions comfortably. Also in-charge of troubleshooting problems for them. 
% \end{itemize}

\vspace{0.5mm}
\header{Experience}


{\textbf{Thapar Institute of Engineering and Technology}}\hfill June 2023 - Present \\
 {\textit{Research Intern under prof. (Dr.) Sachin Kansal }}  \
\vspace{-2.5mm}
\begin{itemize} 
\item Developed an efficient pipeline for extracting frames from video data, utilising advanced image processing techniques, parallel processing, and error handling. 
\vspace{-2mm}
\item Optimised the pipeline's performance, enabling swift extraction of frames from large-scale video datasets while ensuring accuracy and data integrity.
\end{itemize}
{\textbf{Thapar Institute of Engineering and Technology}}\hfill June 2023 \\
 {\textit{Guest Lecturer}}  \
\vspace{-1.5mm}
\begin{itemize} 
\item Invited as guest of  honour to deliver the lecture in Thapar summer School of Machine learning and deep Learning and delivered 3 lectures on Image Processing
\end{itemize}

\header{Projects/Achievements}

{\textbf{PixelMind AI}} \hfill May 2023 \\
\vspace{-1.5mm}
\begin{itemize} 
	\item Paradox’23 in Indian Institute of Technology Madras organised a contest in thhe domain of Image enhancement for students and industry professionals where i won 2nd Position and won a cash Prize of Rs.50K.
    \vspace{-2mm}
    \item Used SRGAN to enhance the pixel resolutions and then used various gaussian filter to enhance the quality of image.
\end{itemize}
{\textbf{Kaggle 2X- Expert}} \hfill June 2023 \\
\vspace{-1mm}
\url{https://www.kaggle.com/eishkaran} \\
\vspace{-2mm}
\begin{itemize} 
	\item On my Kaggle Profile, I have demonstrated my ability to analyse datasets and create effective solutions.By leveraging my skills in exploratory data analysis and algorithm tutorials. I have established myself as a 2x-Kaggle expert and I am committed to continuing to hone my skills in this field.I have been awarded with 16 bronze medals till now
    \vspace{-2mm}
    % \item By leveraging my skills in exploratory data analysis and algorithm tutorials. I have established myself as a 2x-Kaggle expert and I am committed to continuing to hone my skills in this field 
    % \vspace{-2mm}
    \item BirdCLEF 2023 (Organised by Cornell Lab of Ornithology on Kaggle Platform) Got a rank of 169 out of 1189 teams participating all across the globe and was in top 15\%  in which I predicted the sounds of the birds using the PyTorch and applied various DL algorithms 

    
\end{itemize}
{\textbf{CycleGan }} \hfill June 2023\\
\vspace{-1mm}
\url{https://github.com/EISHKARAN/CycleGan} \\
\vspace{-2mm}
\begin{itemize} 
    \item Utilising my expertise in PyTorch and TensorFlow, I made and implemented a robust CycleGAN solution for generating high-quality Monet-style images from ordinary photographs and ranked 19th out of 39 teams
    \vspace{-2mm}
    \item Used the CycleGan generator along with U-net which consisted of a sequence of downsampling blocks followed by a sequence of upsampling blocks
\end{itemize}
{\textbf{Stock Prices Prediction}} \hfill June 2023 \\
\vspace{0.5mm}
\url{https://github.com/EISHKARAN/Stock-Prediction-using-LSTM} \\
\vspace{-2mm}
\begin{itemize} 
	\item Used Yahoo Finance dataset, to create visualisations for Apple, Google, Microsoft and Amazon stock data
    \vspace{-2mm}
    \item Developed a stock predictor using an LSTM model to forecast future stock prices.
\end{itemize}
{\textbf{Blogs }} \hfill April 2023\\
\vspace{-1mm}
\url{https://medium.com/@eishkaransingh} \\
\vspace{-2mm}
\begin{itemize} 
    \item I have authored concise and insightful technical blog posts on machine learning and statistics for data science, fostering knowledge sharing and intellectual growth within the community.
\end{itemize}
\header{Skills}
\vspace{2mm}
\begin{tabular}{ l l }
	Programming Languages: & Python, C++ ,C  \\
	\vspace{1mm}
	Database Management Systems: & MySQL , Oracle \\ 
	Web Development:       & HTML , CSS, JavaScript   \\
	Libraries / Frameworks: & Pandas, Numpy, Seaborn, Tensorflow, Pytorch, XGBoost, scikit-learn, OpenCv        \\
	Machine Learning:             & Regression, Classification, Clustering, Neural Networks, Deep Learning\\
	Soft Skills:           &  Public Speaking, Leadership  \\
	
\end{tabular}


\vspace{2mm}
\header{MISCELLANEOUS}
{\textbf{Career Certificates }} \hfill Jan 2023\\  \\Python, MySQL, Pandas, Supervised Machine Learning\\ 
{\textbf{Silver Zone Olympiad }} \hfill September 2020\\  \\International Informatics Olympiad\\ 
\end{document}
